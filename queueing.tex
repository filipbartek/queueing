% Deadline: end of the week (2014-06-08)

% Assignment:
% https://onedrive.live.com/view.aspx?cid=AF890009B90D6372&resid=AF890009B90D6372%211402&app=Word

% Sage LaTeX:
% https://cloud.sagemath.com/help#help-latex
% http://www.sagemath.org/doc/tutorial/sagetex.html

% Sources:
% http://en.wikipedia.org/wiki/Queueing_theory
% http://en.wikipedia.org/wiki/M/M/1_queue
% http://en.wikipedia.org/wiki/M/M/c_queue
% http://en.wikipedia.org/wiki/Geometric_series#Sum

% Software:
% http://www.moreno.marzolla.name/software/queueing/
% http://octave.sourceforge.net/queueing/index.html
% http://www.gnu.org/software/octave/doc/interpreter/Using-Packages.html#Using-Packages
% http://jmt.sourceforge.net/

\documentclass{article}

\usepackage[utf8]{inputenc}
\usepackage[T1]{fontenc}
\usepackage[english]{babel}

\title{Stationary analysis of one channel queueing systems}
\author{Filip Bártek}

\bibliographystyle{plain}

\usepackage{amsmath}
\usepackage{amssymb}
\usepackage{sagetex}
\usepackage{url}

\usepackage{hyperref} % last in preamble
\begin{document}
\maketitle

% Assignment:
% Analytical solution of queuing systems (QS)
% a) QS with one channel of limited length range of queue
% b) QS with one channel of unlimited length range of queue

\section{Introduction}

In this text, I will analyze some properties of M/M/1 queueing systems.

Using Kendall's notation\cite{wiki:kendall},
an M/M/c queueing system consists of:

\begin{itemize}
\item
a memoryless arrival process,
modeled as Poisson process with rate $\lambda$, and
\item
$c$ servers,
each with a memoryless service time distribution,
modeled as exponential distribution with rate $\mu$.
\end{itemize}

We will only analyze systems with $c = 1$, i.e.~one server.

Another parameter of the queueing systems we will consider
is their capacity,
i.e.~the maximal number of customers allowed in the system.
Let's denote this parameter as $K$.

Altogether, we have three parameters of interest:

\begin{itemize}
\item $K$: capacity
\item $\lambda$: arrival rate
\item $\mu$: service rate
\end{itemize}

The state of the queueing system is the number of customers
currently in the system (including those being served at the moment
and those waiting in the queue). This means that a queueing system has
exactly $K$ states.

In every time $t \geq 0$, the queueing system is in one of the states.
As time $t$ increases, the system may change state according to arrival
process and service process. The number of customers in the system
increases by one when new customer arrives (unless the capacity is full)
and decreases by one when a customer service finishes.

In the remainder of this text,
we will perform stationary analysis of the system,
i.e.~analyze the system in limit time ($t \rightarrow \infty$).

\section{Analysis}

In this section, we will derive the stationary state distribution
and expected state of a queuing system
parametrized by $K$, $\lambda$ and $\mu$.
We will consider both finite and infinite $K$.

\subsection{Distribution of states}

%Let's suppose that the probability distribution of states converges.

Let $\pi_i$ denote the probability that the queueing system
is in state $i$ in limit time ($t \rightarrow \infty$).

The following equation must hold for each pair of states
$i, i+1$ ($i < K$):

\begin{equation}
\pi_i \lambda = \pi_{i+1} \mu
\end{equation}

The equation connects pairs of neighboring states.

Rewriting this equation we get a formula we can use to calculate
$\pi_{i+1}$ from $\pi_i$:

\begin{equation}
\pi_{i+1} = \pi_i \frac{\lambda}{\mu}
\end{equation}

Let $\rho$ denote $\frac{\lambda}{\mu}$.
(We will continue to use this notation in the remainder of the text.)

Using induction, we get a formula that describes $\pi_i$
as a function of $\pi_0$, $\rho$ and $i$:

\begin{equation} \label{eqn:piifrompi0}
\pi_i = \pi_0 \rho^i
\end{equation}

This gives us a system of $K$ equations that describe
$\pi_i$ as a function of $\pi_0$, $i$ and $\rho$
for every $i \in \{1, \ldots, K\}$.

To finish the analysis, we need to restrict the sum of probabilities:

\begin{equation}
\sum_{i=0}^K \pi_i = 1
\end{equation}

Using formula \ref{eqn:piifrompi0}, we get:

\begin{align}
\sum_{i=0}^K \pi_0 \rho^i &= 1 \\
\pi_0 \sum_{i=0}^K \rho^i &= 1 \\
\pi_0 &= \left( \sum_{i=0}^K \rho^i \right) ^{-1}
\end{align}

We use formula \ref{eqn:piifrompi0} again to obtain a general equation:

\begin{align}
\pi_i &= \left( \sum_{i=0}^K \rho^i \right)^{-1} \rho^i
\label{eqn:piifromsum}
\end{align}

This is a general formula that holds for all values of $K$ and $\rho$.
In the following sections,
we will express the sum $\sum_{i=0}^K \rho^i$
to obtain nicer formulas separately
for various cases of $K$ and $\rho$.

\subsubsection{Limited capacity}

Let $K \in \mathbb{N}$.

Let's first consider the case $\rho = 1$ (i.e.~$\lambda = \mu$).

\begin{equation}
\sum_{i=0}^K \rho^i = \sum_{i=0}^K 1 = K + 1
\end{equation}

\begin{align}
\pi_i &= \left( \sum_{i=0}^K \rho^i \right)^{-1} \rho^i \\
&= (K+1)^{-1} 1 \\
&= \frac{1}{K+1}
\end{align}

This means that if arrival and service rates are equal and
the queue is limited, the stationary distribution of states is uniform.

Now let's consider $\rho \neq 1$.
Since $\rho^i$ as a function of $i \in \{0, \ldots, K\}$
is a geometric series, we can use
a formula\cite[section Formula]{wiki:geometric} to express the sum:

\begin{equation}
\sum_{i=0}^K \rho^i = \frac{1 - \rho^{K+1}}{1 - \rho}
\end{equation}

Replacing the sum in formula \ref{eqn:piifromsum} we get:

\begin{align}
\pi_i &= \left( \sum_{i=0}^K \rho^i \right)^{-1} \rho^i \\
&= \left( \frac{1 - \rho^{K+1}}{1 - \rho} \right)^{-1} \rho^i \\
&= \frac{1 - \rho}{1 - \rho^{K+1}} \rho^i
\end{align}

Namely:

\begin{equation} \label{eqn:pi0limited}
\pi_0 = \frac{1 - \rho}{1 - \rho^{K+1}}
\end{equation}

\subsubsection{Unlimited capacity}

Let $K = \infty$.

If $\rho \geq 1$, the queue grows indefinitely
and the system doesn't have a stationary distribution.

Let $\rho < 1$.

We can express the sum $\sum_{i=0}^K \rho^i = \sum_{i=0}^\infty \rho^i$
using polylogarithm of order $0$:

\begin{align}
\sum_{i=0}^\infty \rho^i &= 1 + \sum_{i=1}^\infty \rho^i \\
&= 1 + Li_0(\rho) \\
&= 1 + \frac{\rho}{1 - \rho} \label{eqn:li0} \\
&= \frac{(1 - \rho) + \rho}{1 - \rho} \\
&= \frac{1}{1 - \rho}
\end{align}

In equation \ref{eqn:li0} we used expression of $Li_0$ shown in
\cite[section Particular values]{wiki:polylogarithm}.

Replacing the sum in formula \ref{eqn:piifromsum} we get:

\begin{align}
\pi_i &= \left( \sum_{i=0}^K \rho^i \right)^{-1} \rho^i \\
&= \left( \frac{1}{1 - \rho} \right)^{-1} \rho^i \\
&= (1 - \rho) \rho^i
\end{align}

This shows that the distribution of states is geometric
with success probability $1 - \rho$.

Namely:

\begin{align}
\pi_0 &= 1 - \rho \label{eqn:pi0unlimited}
\end{align}

\subsection{Expected number of customers}

Let $E[I]$ denote the expected number of customers (i.e.~state)
of the queueing system in limit time.

\begin{align}
E[I] &= \sum_{i=0}^K{i \pi_i} \\
&= \sum_{i=0}^K{i \pi_0 \rho^i} \\
&= \pi_0 \sum_{i=0}^K{i \rho^i} \label{eqn:ei}
\end{align}

Again, we will discuss cases of finite and infinite $K$ separately.

\subsubsection{Limited capacity}

Let $K \in \mathbb{N}$.

Let $\rho = 1$.

\begin{align}
E[I] &= \pi_0 \sum_{i=0}^K{i \rho^i} \\
&= \frac{1}{K+1} \sum_{i=0}^K{i} \\
&= \frac{1}{K+1} \frac{(K+1)K}{2} \\
&= \frac{K}{2}
\end{align}

Let $\rho \neq 1$.

\begin{align}
\sum_{i=0}^K{i \rho^i}
&= \sum_{i=1}^K{i \rho^i} \\
&= \rho \frac{1 - (K+1) \rho^K + K \rho^{K+1}}{(1 - \rho)^2}
\label{eqn:sumirhoi}
\end{align}

Expression \ref{eqn:sumirhoi} comes from \cite[section Low-order polylogarithms]{wiki:series}.

Using equation \ref{eqn:ei} and the value of $pi_0$ from equation
\ref{eqn:pi0limited} we get:

\begin{align}
E[I]
&= \pi_0 \sum_{i=0}^K{i \rho^i} \\
&= \frac{1 - \rho}{1 - \rho^{K+1}}
\rho \frac{1 - (K+1) \rho^K + K \rho^{K+1}}{(1 - \rho)^2} \\
&= \rho \frac{1 - (K+1) \rho^K + K \rho^{K+1}}{(1 - \rho)(1 - \rho^{K+1})}
\end{align}

\subsubsection{Unlimited capacity}

Let $K = \infty$.

If $\rho \geq 1$, the queue grows indefinitely
and the system doesn't have a stationary distribution.

Let $\rho < 1$.

We can express the sum
$\sum_{i=0}^K{i \rho^i} = \sum_{i=0}^\infty{i \rho^i}$
using polylogarithm of order $-1$:

\begin{align}
\sum_{i=0}^\infty{i \rho^i} &= \sum_{i=1}^\infty{i \rho^i} \\
&= Li_{-1}(\rho) \\
&= \frac{\rho}{(1 - \rho)^2} \label{eqn:li-1}
\end{align}

In equation \ref{eqn:li-1} we used expression of $Li_{-1}$ shown in \cite[section Particular values]{wiki:polylogarithm}.

Applying equations \ref{eqn:pi0unlimited} and \ref{eqn:li-1}
in equation \ref{eqn:ei} we get:

\begin{align}
E[I] &= \pi_0 \sum_{i=0}^K{i \rho^i} \\
&= (1 - \rho) \frac{\rho}{(1 - \rho)^2} \\
&= \frac{\rho}{1 - \rho}
\end{align}

\section{Conclusions}

We have derived the following expressions for
stationary state probability distribution and
expected state of single channel queueing systems:

\begin{tabular}{c | c || c | c | c}
$K$ & $\rho$ & $\pi_i$ & $\pi_0$ & $E[I]$ \\
\hline
$K \in \mathbb{N}$ & $\rho = 1$ & $\frac{1}{K + 1}$
& $\frac{1}{K + 1}$ & $\frac{K}{2}$ \\
$K \in \mathbb{N}$ & $\rho \neq 1$
& $\frac{1 - \rho}{1 - \rho^{K+1}} \rho^i$
& $\frac{1 - \rho}{1 - \rho^{K+1}}$
& $\rho \frac{1 - (K+1) \rho^K + K \rho^{K+1}}{(1 - \rho)(1 - \rho^{K+1})}$ \\
$K = \infty$ & $\rho \geq 1$
& $0$ & $0$ & $\infty$ \\
$K = \infty$ & $\rho < 1$
& $(1 - \rho) \rho^i$ & $(1 - \rho)$ & $\frac{\rho}{1 - \rho}$ \\
\end{tabular}

\appendix

\section{Sage implementation}

To test the derived expressions on some small parameter values,
I implemented Sage\footnote{\url{http://www.sagemath.org/}} functions \texttt{pi} and \texttt{ei} that compute values of
$\pi_i$ and $E[I]$ respectively.

\begin{sageblock}
import math
def pi(i, rho, K):
    assert i >= 0
    assert rho >= 0
    assert K >= 0
    if math.isinf(K):
        if rho >= 1:
            return 0
        else:
            return (1 - rho) * rho^i
    else:
        if rho == 1:
            return 1 / (K+1)
        else:
            return (1 - rho) * rho^i / (1 - rho^(K+1))
\end{sageblock}
\begin{sagesilent}
assert pi(0, 1/2, 1) == 2/3
assert pi(0, 2, 1) == 1/3
\end{sagesilent}

\begin{sageblock}
import math
def ei(rho, K):
    assert rho >= 0
    assert K >= 0
    if math.isinf(K):
        if rho >= 1:
            return float("inf")
        else:
            return rho / (1 - rho)
    else:
        if rho == 1:
            return K / 2
        else:
            return pi(0, rho, K) * \
                rho * (1 - (K+1)*rho^K + K*rho^(K+1)) / \
                (1 - rho)^2
\end{sageblock}
\begin{sagesilent}
assert ei(1/2, 1) == 1/3
assert ei(2, 1) == 2/3
\end{sagesilent}

\section{Example solutions}

\begin{tabular}{c | c || c | c}
$K$ & $\rho$ & $\pi_0$ & $E[I]$ \\
\hline
\hline
$0$ & $\frac{1}{2}$ & $\sage{pi(0, 1/2, 0)}$ & $\sage{ei(1/2, 0)}$ \\
$1$ & $\frac{1}{2}$ & $\sage{pi(0, 1/2, 1)}$ & $\sage{ei(1/2, 1)}$ \\
$2$ & $\frac{1}{2}$ & $\sage{pi(0, 1/2, 2)}$ & $\sage{ei(1/2, 2)}$ \\
$\infty$ & $\frac{1}{2}$ & $\sage{pi(0, 1/2, float("inf"))}$
& $\sage{ei(1/2, float("inf"))}$ \\
\hline
$0$ & $1$ & $\sage{pi(0, 1, 0)}$ & $\sage{ei(1, 0)}$ \\
$1$ & $1$ & $\sage{pi(0, 1, 1)}$ & $\sage{ei(1, 1)}$ \\
$2$ & $1$ & $\sage{pi(0, 1, 2)}$ & $\sage{ei(1, 2)}$ \\
$\infty$ & $1$ & $\sage{pi(0, 1, float("inf"))}$
& $\sage{ei(1, float("inf"))}$ \\
\hline
$0$ & $2$ & $\sage{pi(0, 2, 0)}$ & $\sage{ei(2, 0)}$ \\
$1$ & $2$ & $\sage{pi(0, 2, 1)}$ & $\sage{ei(2, 1)}$ \\
$2$ & $2$ & $\sage{pi(0, 2, 2)}$ & $\sage{ei(2, 2)}$ \\
$\infty$ & $2$ & $\sage{pi(0, 2, float("inf"))}$
& $\sage{ei(2, float("inf"))}$
\end{tabular}

\bibliography{queueing}

\end{document}
